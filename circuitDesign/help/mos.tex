\section{Matlab files related to the MOS transistor}
\subsection{mosAcmPerimeter}
\label{sec:mosAcmPerimeter}
\index{mosAcmPerimeter@\textsf{mosAcmPerimeter}}
\begin{verbatim}
 MOSACMPERIMETER computation of the perimeter of a source or drain region
 
    mosAcmPerimeter(ACM, LSWS, LSWG) returns the perimeter of a diffusion region
    (= source or drain region). This perimeter is computed according to the area 
    calculation method ACM with the given perimeter of the diffusion region under
    the gate (argument LSWG) and not under the gate (argument LSWS).
 
   (c) IMEC, 2004
   IMEC confidential 
 

\end{verbatim}

\newpage
\subsection{mosAreaPerimeter}
\label{sec:mosAreaPerimeter}
\index{mosAreaPerimeter@\textsf{mosAreaPerimeter}}
\begin{verbatim}
 MOSAREAPERIMETER computation of areas and perimeters of source and drain 
  regions of a MOS transistor.
 
    MOS = mosAreaPerimeter(MOS) computes 
     - area of source and drain regions (fields "ad" and "as"), 
     - length of the sidewall of the source or drain
       area which is not under the gate (fields "lsd" and "lss"), 
     - length of the sidewall of the source or drain area which is under the
       gate (fields "lgd" and "lgs"), 
     - "pstot", which is the sum of "lss" and "lgs",
     - "pdtot", which is the sum of "lsd" and "lgd",
     - "ps" which is either "pstot" or "pstot - w", depending on the model that
       is used.
     - "pd", which is either "pdtot" or "pdtot - w", depending on the model
       that is used. 
     These quantities are added as fields to the given MOS transistor if they
     do not exist yet, else they are not recomputed.
     To compute these quantities, the transistor must have the following
     fields: the width <MOS>.w, length of the drain/source diffusion areas not
     between two poly stripes (fields <MOS>.lsod and <MOS>.lsos, respectively),
     the length of drain/source diffusion areas between two poly stripes
     (fields <MOS>.lsogd and <MOS>.lsogs, respectively),
     the amount of fingers (field <MOS>.nFingers) and the multiplicity
     <MOS>.mult.
 
     EXAMPLES :
 
     example 1:
     Mn1 = mosAreaPerimeter(Mn1)
 
     example 2:
     Mn1.w = 20, Mn1.LSOs = 1, Mn1.LSOd = 1, Mn1.LSOgs = 1, Mn1.LSOgd = 1,
     Mn1.nFingers = 1, Mn1.mult = 1 
     Mn1 = mosAreaPerimeter(Mn1)
     returns Mn1 with the following fields computed
     Mn1.ad = 20, Mn1.lsd = 22, Mn1.lgd = 20, Mn1.as = 20, Mn1.lss = 22, 
     Mn1.lgs = 20, Mn1.pstot = 42, Mn1.pdtot = 42
    
     example 3:    
     Mn1.w = 20, Mn1.LSOs = 1, Mn1.LSOd = 1, Mn1.LSOgs = 1, Mn1.LSOgd = 1,
     Mn1.nFingers = 4, Mn1.mult = 1
     Mn1 = mosAreaPerimeter(Mn1)
     returns Mn1 with the following fields computed
     Mn1.ad = 10, Mn1.lsd = 4, Mn1.lgd = 20, Mn1.as = 15, Mn1.lss = 16,
     Mn1.lgs = 20, Mn1.pstot = 36, Mn1.pdtot = 24
 
 
 
   (c) IMEC, 2004
   IMEC confidential 
 

\end{verbatim}

\newpage
\subsection{mosCheckFreq}
\label{sec:mosCheckFreq}
\index{mosCheckFreq@\textsf{mosCheckFreq}}
\begin{verbatim}
 MOSCHECKFREQ check whether a given operating frequency is higher than fT
   (c) IMEC, 2004
   IMEC confidential 
 

\end{verbatim}

\newpage
\subsection{mosCheckOpParam}
\label{sec:mosCheckOpParam}
\index{mosCheckOpParam@\textsf{mosCheckOpParam}}
\begin{verbatim}
 MOSCHECKOPPARAM checks whether a given STRING is an operating point
    parameter that is supported in the table-based circuit sizing approach.
 
    MOSCHECKOPPARAM(STRING) returns 1 if the given STRING is the name of an 
    operating point parameter that is supported in the table-base circuit sizing
    approach. If the STRING does not correspond to the name of a supported
    parameter, then 0 is returned.
    A given table may contain only a subset of the supported operating
    point parameters.
 
   (c) IMEC, 2004
   IMEC confidential 
 

\end{verbatim}

\newpage
\subsection{mosCheckSaturation}
\label{sec:mosCheckSaturation}
\index{mosCheckSaturation@\textsf{mosCheckSaturation}}
\begin{verbatim}
 MOSCHECKSATURATION check whether a transistor operates in the saturation
     region.
 
     mosCheckSaturation(MOS) gives a warning when transistor MOS is not
     saturated. This function compares MOS.vds to MOS.vdsat if they exist.
     When one of these fields does not exist, an error is signaled.
 
 
 
   (c) IMEC, 2004
   IMEC confidential 
 

\end{verbatim}

\newpage
\subsection{mosCheckWidth}
\label{sec:mosCheckWidth}
\index{mosCheckWidth@\textsf{mosCheckWidth}}
\begin{verbatim}
 MOSCHECKWIDTH comparison of a transistor width to the critical width
 
    result = mosCheckWidth(MOSNAME, WIDTH, TABLE) returns 0 and gives a warning when
    the specified WIDTH of the specified MOS transistor with name MOSNAME is below 
    the critical width of the given TABLE.
    This critical width can be obtained with tableWcrit(TABLE)
 
    See also tableWcrit
 
 
   (c) IMEC, 2004
   IMEC confidential 
 

\end{verbatim}

\newpage
\subsection{mosCompareDcOp}
\label{sec:mosCompareDcOp}
\index{mosCompareDcOp@\textsf{mosCompareDcOp}}
\begin{verbatim}
 MOSCOMPAREDCOP comparison of operating point information computed in matlab 
  and with a circuit simulator
 
   RESULT = mosCompareDcOp(CIRCUIT, DESIGNKIT, SIMULATOR, VERBOSELEVEL,
   FID) returns a cell array in which each element is a cell array with three
   elements, the first element being the name of a parameter of interest (see
   description of the fields of a designkit, see cirDesignkitInit), the element
   being the average and the maximum of the (unsigned) relative errors between
   the value of the 
   parameter under consideration computed in Matlab and with the given SIMULATOR,
   averaged over all n-MOS and p-MOS transistors in the given CIRCUIT. 
   For the time being only spectre is supported.
   The VERBOSELEVEL is either 0, 1 or 2. For values of 1 and 2, some comparison
   data is written into a given file identifier FID.
 
 
   EXAMPLE:
 
   a = mosCompareDcOp(circuit, designkit, 'spectre', 1, 1);
 
  Then a{1} can be an array that looks like
  a{1}
 
  ans = 
 
   Columns 1 through 5
 
     'idsnmos'    'vgsnmos'    'vdsnmos'    'vsbnmos'    'gmnmos'
 
   Columns 6 through 10
 
     'gmbsnmos'    'gdsnmos'    'vthnmos'    'vdsatnmos'    'ftnmos'
 
   Columns 11 through 15
 
     'idspmos'    'vgspmos'    'vdspmos'    'vsbpmos'    'gmpmos'
 
   Columns 16 through 20
 
     'gmbspmos'    'gdspmos'    'vthpmos'    'vdsatpmos'    'ftpmos'
 
  Further, a{2} is a cell array with the average rel. error values (averaged
  over all n-MOS transistors and all p-MOS transistors) and a{3} with the
  maximum value of the relative error.  
 
   (c) IMEC, 2004
   IMEC confidential 
 

\end{verbatim}

\newpage
\subsection{mosDiodeVth}
\label{sec:mosDiodeVth}
\index{mosDiodeVth@\textsf{mosDiodeVth}}
\begin{verbatim}
  computation of vth of a diode connected MOS transistor without body
  effect.
  This is done with a fixed point iteration
 
   (c) IMEC, 2004
   IMEC confidential 
 

\end{verbatim}

\newpage
\subsection{mosExtValue}
\label{sec:mosExtValue}
\index{mosExtValue@\textsf{mosExtValue}}
\begin{verbatim}
 MOSEXTVALUE computation of parameters related to the extrinsic 
 part of a MOS transistor.
 
     value = mosExtValue(NAME, MOS) returns the value of a parameter related
     to the extrinsic part of a MOS transistor. The NAME of the parameter
     should be specified as a string. Possible names are
     cdbE, csbE, as, ad, ps, pd, lsd, lgd, lss, lgs
 
     EXAMPLE :
 
        Mn1.cdbE = mosExtValue('cdbE', Mn1);
 
 
 
   (c) IMEC, 2004
   IMEC confidential 
 

\end{verbatim}

\newpage
\subsection{mosIntValue}
\label{sec:mosIntValue}
\index{mosIntValue@\textsf{mosIntValue}}
\begin{verbatim}
 MOSINTVALUE gets the value of a intrinsic MOS parameter 
 
     PARAMVALUE = mosIntValue(ELECPARAM, MOS) returns the
     value of a MOS operating point parameter ELECPARAM (specified as a string) for
     transistor MOS. The value is computed
     in S.I. units.   
     ELECPARAM must be included in the table of operating point parameter values
     of a transistor. Possible names for ELECPARAM can be found by running
     tableDisplay(mosTable(mos))  or a parameter with name PARAM can be
    checked on its validity using mosCheckOpParam(PARAM).                                           
 
    See also tableDisplay, tableValueWref, mosWidth, tableWref, tableWcrit
    
 
     EXAMPLE :
 
       ids = mosIntValue('ids', Mn1)
 
 
   (c) IMEC, 2004
   IMEC confidential 
 

\end{verbatim}

\newpage
\subsection{mosIntValueWref}
\label{sec:mosIntValueWref}
\index{mosIntValueWref@\textsf{mosIntValueWref}}
\begin{verbatim}
 MOSINTVALUEWREF retrieving value of an intrinsic MOS parameter for 
 a given MOS transistor, assuming its width equals the reference width.
                                                                           
    RESULT = mosIntValueWref(PARAM, MOS) returns the value of the 
    MOS operating point parameter PARAM (specified as a string) for a given MOS
    transistor.   
    PARAM must be included in the TABLE of intrinisc values of a transistor
    of the same type of the given transistor. Possible names for PARAM can be
    found by running tableDisplay(TABLE) or a parameter with name PARAM can be
    checked on its validity using mosCheckOpParam(PARAM).
 
    EXAMPLE :                                                                
 
       id = mosIntValueWref('ids', Mn1)                            
   
    See also tableDisplay, mosIntValue, mosWidth, tableWref,
    tableValueWref, mosCheckOpParam
    
 
 
  The value of the parameter is computed by linear interpolation in the 
  argument table.
 
   (c) IMEC, 2004
   IMEC confidential 
 

\end{verbatim}

\newpage
\subsection{mosJuncap}
\label{sec:mosJuncap}
\index{mosJuncap@\textsf{mosJuncap}}
\begin{verbatim}
 MOSJUNCAP computation of junction capacitance of a source or drain region 
 of a MOS transistor.
 
     cJunction = mosJuncap(TABLE, VDIFFBULK, AREA, PERIMTOT, W) returns
     the value of the junction capacitance of a source or drain region for a
     given type of a MOS transistor, specified by the given TABLE. For the
     junction one must specify the AREA, the total perimeter PERIMTOT of the
     drain or source region, the gate width W and the voltage over the
     junction VDIFFBULK. The latter is the voltage difference between the
     source or drain region and the bulk (should be positive for an n-MOS and 
     negative for a p-MOS) 
 
     EXAMPLE :
 
        for a MOS transistor with one finger, a width of 2 micrometer, a
        length of 1 micrometer for the diffusion areas, and 1 Volt over the
        junction between the diffusion area of the bulk and a table name N, the
        function call is
        cdb = mosJuncap(N, 1, 2e-12, 6e-6, 2e-6);
 
 
 
   (c) IMEC, 2004
   IMEC confidential 
 

\end{verbatim}

\newpage
\subsection{mosLayoutLengths}
\label{sec:mosLayoutLengths}
\index{mosLayoutLengths@\textsf{mosLayoutLengths}}
\begin{verbatim}
 MOSLAYOUTLENGTHS computation of lengths of the source/drain diffusion between
    and not between two poly stripes.
 
  MOS = mosLayoutLengths(MOS, TABLE) computes for the given MOS transistor the
  lengths of the source/drain diffusion between and not between two poly
  stripes. This computation is done based on layout rule information, that is
  stored in the given TABLE, that corresponds to the given MOS transistor. The
  fields lsos, lsod, lsogs and lsogd of the given MOS transistor are filled
  with 
  - the value of the length of the source diffusion not between two poly 
  stripes (field "lsos"), 
  - the value of the length of the drain diffusion not between two poly
  stripes (field "lsod"),
  - the value of the length of the source diffusion between two poly 
  stripes (field "lsogs"),
  - the value of the length of the drain diffusion between two poly 
  stripes (field "lsogd")
  Finally, the MOS transistor is returned.
  The following layout rule data should be present in the field table.Info:
  - table.Info.wco: width of a contact
  - table.Info.eactco: extension of active over contact
  - table.Info.dcopss: distance between contact and poly stripe in the source
  area 
  - table.Info.dcopsd: distance between contact and poly stripe in the drain
  area 
 
   (c) IMEC, 2004
   IMEC confidential 
 

\end{verbatim}

\newpage
\subsection{mosLsoD}
\label{sec:mosLsoD}
\index{mosLsoD@\textsf{mosLsoD}}
\begin{verbatim}
 MOSLSOS computation of length of the drain diffusion (not between two poly
     stripes)
 
  lsod = mosLsoD(WCO, EACTCO, DCOPSD) computes the length of the drain
  diffusion not between two poly stripes as a function of contact width WCO,
  the extension of active over contact EACTCO, and the distance between contact
  and poly stripe in the drain area DCOPSD
 
 
   (c) IMEC, 2004
   IMEC confidential 
 

\end{verbatim}

\newpage
\subsection{mosLsoGd}
\label{sec:mosLsoGd}
\index{mosLsoGd@\textsf{mosLsoGd}}
\begin{verbatim}
 MOSLSOGD computation of length of the drain diffusion between two poly
     stripes
 
  lsogd = mosLsoGd(WCO, DCOPSD) computes the length of the drain
  diffusion between two poly stripes as a function of contact width WCO and
  the distance between contact and poly stripe in the drain area DCOPSD
 
 
   (c) IMEC, 2004
   IMEC confidential 
 

\end{verbatim}

\newpage
\subsection{mosLsoGs}
\label{sec:mosLsoGs}
\index{mosLsoGs@\textsf{mosLsoGs}}
\begin{verbatim}
 MOSLSOGS computation of length of the source diffusion between two poly
     stripes
 
  lsogs = mosLsoGs(WCO, DCOPSS) computes the length of the source
  diffusion between two poly stripes as a function of contact width WCO and
  the distance between contact and poly stripe in the source area DCOPSS
 
 
   (c) IMEC, 2004
   IMEC confidential 
 

\end{verbatim}

\newpage
\subsection{mosLsoS}
\label{sec:mosLsoS}
\index{mosLsoS@\textsf{mosLsoS}}
\begin{verbatim}
 MOSLSOS computation of length of source diffusion (not between two poly
     stripes)
 
  lsos = mosLsoS(WCO, EACTCO, DCOPSS) computes the length of the source
  diffusion not between two poly stripes as a function of contact width WCO,
  the extension of active over contact EACTCO, and the distance between contact
  and poly stripe in the source area DCOPSS
 
 
   (c) IMEC, 2004
   IMEC confidential 
 

\end{verbatim}

\newpage
\subsection{mosName}
\label{sec:mosName}
\index{mosName@\textsf{mosName}}
\begin{verbatim}
 MOSNAME returns the name of a given MOS transistor as a string
 
 
     NAME = mosName(MOS) returns a string that contains the name of the
     given MOS transistor.
 
     EXAMPLE :
   
       nameMn1 = mosName(Mn1);
 
 
   (c) IMEC, 2004
   IMEC confidential 
 

\end{verbatim}

\newpage
\subsection{mosNfingers}
\label{sec:mosNfingers}
\index{mosNfingers@\textsf{mosNfingers}}
\begin{verbatim}
 MOSNFINGERS computation of the number of fingers for a MOS transistor.
 
 
  MOS = mosNfingers(MOS) computes the number of
  fingers for a MOS transistor if the field MOS.nFingers does not exist. If the
  field exists, then nothing is changed. If the field does not exist, then the
  computed number of fingers is stored in the field MOS.nFingers. 
  This function can only work if the fields MOS.w and MOS.geo exist. The
  function always returns an even or odd  number of fingers, depending on the
  value of MOS.geo. 
  The function tries to find a fingerwidth between a minimum and a maximum
  width. These widths are determined as follows: if the fields
  MOS.minFingerWidth and MOS.maxFingerWidth exist, then these values are used
  for minimum and maximum width, respectively. If MOS.minFingerWidth does not
  exist, then the value of Wcrit from the table MOS.table is taken. If
  MOS.maxFingerWidth does not exist, then the value maxFingerWidth is taken
  from the base. If the latter does not exist, then an error is given. 
 
  Example:
 
  Mn1 = mosNfingers(Mn1);
 
 
   (c) IMEC, 2004
   IMEC confidential 
 

\end{verbatim}

\newpage
\subsection{mosNsquares}
\label{sec:mosNsquares}
\index{mosNsquares@\textsf{mosNsquares}}
\begin{verbatim}
 MOSNSQUARES computation of number of squares of source and drain 
  diffusion for the calculation of extrinsic source and drain resistances of a
  MOS transistor. 
 
    MOS = mosNsquares(MOS) computes the number of squares of source and drain 
  diffusion. These quantities are added as the fields "nrd" (# squares of drain
  diffusion) and "nrs" (idem for source diffusion), respectively, if at least
  these fields do not exist yet. 
  To compute these quantities, the transistor must have the following
  fields: the width <MOS>.w, the length of drain/source diffusion areas between
  two poly stripes (fields <MOS>.lsogd and <MOS>.lsogs, respectively).
 
 
   (c) IMEC, 2004
   IMEC confidential 
 

\end{verbatim}

\newpage
\subsection{mosOpValues}
\label{sec:mosOpValues}
\index{mosOpValues@\textsf{mosOpValues}}
\begin{verbatim}
 MOSOPVALUES computation of all operating parameters of a MOS transistor
 
    MOS = mosOpValues(MOS) computes the value of intrinsic and extrinsic
    operating point parameters (in S.I. units) of a given transistor MOS. The
    datastructure of the transistor which is specified as an argument is also
    returned. However, after this function has returned, all fields of the
    transistor related to intrinsic and extrinsic operating point parameters
    have been added to the datastructure of the transistor. Existing extrinsic
    geometry parameters are NOT recomputed. Also, the
    values of vgb, vgd and vdb are computed (or updated) from vgs, vds and vsb. 
                                                                           
    EXAMPLE :                                                                
 
        Mn1 = mosOpValues(Mn1)                          
 
 
 
   (c) IMEC, 2004
   IMEC confidential 
 

\end{verbatim}

\newpage
\subsection{mosPrintSizesAndOpInfo}
\label{sec:mosPrintSizesAndOpInfo}
\index{mosPrintSizesAndOpInfo@\textsf{mosPrintSizesAndOpInfo}}
\begin{verbatim}
  MOSPRINTSIZESANDOPINFO print sizes and operating point information of MOS
  transistors in a circuit and its subcircuits.
 
    mosPrintSizesAndOpInfo(FID, CIRCUIT) prints the sizes and operating point
    information of the transistors in the given CIRCUIT and its subcircuits to
    the given file identifier FID. Two file identifiers are automatically
    available and need not be opened.  They are FID=1 (standard output) and
    FID=2 (standard error).
 
   See also fopen.
 
 
   (c) IMEC, 2004
   IMEC confidential 
 

\end{verbatim}

\newpage
\subsection{mosTable}
\label{sec:mosTable}
\index{mosTable@\textsf{mosTable}}
\begin{verbatim}
 MOSTABLE returns the table of operating point parameters from which the
 intrinsic operating point of a MOS transistor are derived.
 
     TABLE = mosTable(MOS) returns the table (this is a structure) from
     which the intrinsic operating point parameters of the given MOS
     transistor are derived. 
 
   (c) IMEC, 2004
   IMEC confidential 
 

\end{verbatim}

\newpage
\subsection{mosVsbBody}
\label{sec:mosVsbBody}
\index{mosVsbBody@\textsf{mosVsbBody}}
\begin{verbatim}
 MOSVSBBODY computation of the source voltage of a MOS transistor with body
 effect
 
     VSB = mosVsbBody(MOS, VGB, VDB, VOV, ESTIMATE) determines the
     source-bulk voltage VSB of a MOS transistor by using the given TABLE. 
     A specified value for the overdrive voltage VOV is given as an argument, 
     as well as a value for the gate-bulk voltage VGB and drain-bulk voltage VDB. 
 
     The voltage VSB is determined by fixed-point iteration. A start value
     ESTIMATE needs to be given for this iteration.
 
    EXAMPLE:
     
       Mn1.vsb = mosVsbBody(Mn1, vdd/2, vdd, 0.2, vdd/2 - 0.2);
 
 
 
   (c) IMEC, 2004
   IMEC confidential 
 

\end{verbatim}

\newpage
\subsection{mosWidth}
\label{sec:mosWidth}
\index{mosWidth@\textsf{mosWidth}}
\begin{verbatim}
 MOSWIDTH returns MOS transistor width for a given operating point parameter 
    
    W  = mosWidth(ELECPARAM, VALUE, MOS) returns the width W of a MOS
    transistor. This returned width W is such that a wanted value VALUE for
    the MOS operating point parameter ELECPARAM (specified as a string) is
    realized. ELECPARAM must be included in the TABLE of a transistor of the same
    type. Possible names for PARAM can be found by running tableDisplay on
    mosTable(mos). If the returned width is below the critical width a warning 
    is given. If the computed width is below the minimum width for the given 
    technology, then an error is given.
 
    See also mosTable, tableDisplay, tableWcrit, tableWmin, tableWref
 
    EXAMPLE :
 
      Mn1.w = mosWidth('ids', 0.001, Mn1);
 
      In this example, the width of Mn1 is computed as Mn1.w, based on a
      specification of 1mA for the drain current of Mn1. It is assumed that the
      terminal voltages and the length of Mn1 are already determined.
 
 
 
   (c) IMEC, 2004
   IMEC confidential 
 

\end{verbatim}

\newpage
\subsection{mosWriteParams}
\label{sec:mosWriteParams}
\index{mosWriteParams@\textsf{mosWriteParams}}
\begin{verbatim}
 MOSWRITEPARAMS writes parameters that have been determined in Matlab to a
   the fields of a structure parStruct
 
   (c) IMEC, 2004
   IMEC confidential 
 

\end{verbatim}

\newpage
\newpage
