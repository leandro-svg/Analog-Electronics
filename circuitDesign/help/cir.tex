\section{Matlab files related to the circuit level}
\subsection{cirAllowedPrintFields}
\label{sec:cirAllowedPrintFields}
\index{cirAllowedPrintFields@\textsf{cirAllowedPrintFields}}
\begin{verbatim}
 CIRALLOWEDPRINTFIELDS returns a list names of the fields of a circuit
     element of type ELTYPE, that can be dumped into/modified in a circuit
     simulation file. 
 
     See also cirInit
 
     EXAMPLE:
 
     cirAllowedPrintFields(circuit, 'nmos')
 
     This returns the list {'w', 'lg', 'ad', 'as', 'pd', 'ps', ...
     'nFingers', 'lsos', 'lsod', 'lsogs', 'lsogd', 'mult'}
 
   (c) IMEC, 2004
   IMEC confidential 
 

\end{verbatim}

\newpage
\subsection{cirChangePrintNames}
\label{sec:cirChangePrintNames}
\index{cirChangePrintNames@\textsf{cirChangePrintNames}}
\begin{verbatim}
 CIRCHANGEPRINTNAMES adapts the list of fields that will be written into a 
     circuit simulation file.
 
     cirChangePrintNames(CIRCUIT, ELTYPE, NAMELIST, PRINTNAMELIST) modifies two
     cell arrays:
     1. the one that contains the parameters of a circuit element of a given
     type ELTYPE (must be a string) is set equal to the argument NAMELIST;
     2. the one that contains for each of these parameters the string that is
     written into the simulation file, is set equal to the argument
     PRINTNAMELIST. paramFields and the paramPrintNames, which
     are strings that are the fieldnames of a circuit element of type ELTYPE
     when this circuit element is defined with cirElementDef.
 
     This function changes the fields circuit.defaults.(eltype).fieldValues{5}
     and circuit.defaults.(eltype).fieldValues{6}. If this function is called,
     it must be done before any circuit element is defined.
 
     See also cirInit, cirElementDef, cirDefaultValues. 
 
     EXAMPLE:
     
    cir = cirChangePrintNames(cir, 'nmos', {'w', 'lg', 'nFingers'}, ...
      {'Wnom', 'Lnom', 'FOLD'})
 
    With this example, we determine for circuit "cir" that in the circuit simulation file which
    will be written or modified, the width (w), the length (lg) and the number
    of fingers of an n-MOS will be written/modified. For example, in a circuit
    netlist (Spectre format) that contains n-MOS transistor Mn1, for which we
    have computed in Matlab w=10e-6, lg=0.2e-6 and nFingers = 5, the above
    example would yield a line in the circuit simulation file of the form
 
    Mn1 (drain gate source bulk) modelname Wnom=10e-6 Lnom=0.2e-6 FOLD=5
 
 
   (c) IMEC, 2004
   IMEC confidential 
 

\end{verbatim}

\newpage
\subsection{cirCheckEltype}
\label{sec:cirCheckEltype}
\index{cirCheckEltype@\textsf{cirCheckEltype}}
\begin{verbatim}
 CIRCHECKELTYPE check whether the specified string is a valid element type.
 
   cirCheckEltype(STRING) returns nothing if the string corresponds to a valid
   element type. Otherwise, an error is signaled. Valid element types are
   'nmos', 'pmos', 'res' (for resistor), 'cap' (for capacitor), 'v' (for
   voltage source), 'i' (for current source) and subcircuit ('subckt').
   These names are case sensitive.
 
   EXAMPLE :
  
       checkEltype('nmos')
          returns nothing, since the element type exists.
       checkEltype('NMOS')
           returns an error.
 
  See also cirInit
 
 
 
   (c) IMEC, 2004
   IMEC confidential 
 

\end{verbatim}

\newpage
\subsection{cirCheckInChoice}
\label{sec:cirCheckInChoice}
\index{cirCheckInChoice@\textsf{cirCheckInChoice}}
\begin{verbatim}
 CIRCHECKINCHOICE defines every field of a structure of design choices as a ...
     variable in the base workspace and copies this structure to a field of 
     the given circuit
 
    CIRCUIT = CIRCHECKINCHOICE(CIRCUIT, CHOICE) adds the field CHOICE to the
    given CIRCUIT. These fields have the same contents as in the variable
    CHOICE. Further, each field of CHOICE is assigned to new variables in the
    base workspace. The name of each new variable is the name of the
    corresponding field.
 
    Examples:
 
    Assume there is a structure "choice" in the base workspace, with
      choice.Vdd.val = 1;
      choice.Mn1.vgs = 0.5;
    while there exists a circuit "ota".
    Then a call like
 
    ota = cirCheckInChoice(ota, choice)
 
    will have the following effects:
    1. the variable "ota" has a new field "choice" and
       ota.choice.Vdd.val = 1;
       ota.choice.Mn1.vgs = 0.5;
    2. new variables are created, namely
       Vdd.val 
       Mn1.vgs
       Their value is 1 and 0.5, respectively.
 
 
   (c) IMEC, 2004
   IMEC confidential 
 

\end{verbatim}

\newpage
\subsection{cirDefaultNames}
\label{sec:cirDefaultNames}
\index{cirDefaultNames@\textsf{cirDefaultNames}}
\begin{verbatim}
 CIRDEFAULTNAMES lists the fields names that are initialized when a circuit 
 element is defined.
 
     LIST = cirDefaultNames(CIRCUIT, ELTYPE) returns a cell array of
     strings that are the fieldnames of a circuit element of type ELTYPE
     when this circuit element is defined with cirElementDef.
 
     See also cirInit, cirElementDef, cirDefaultValues. 
 
   (c) IMEC, 2004
   IMEC confidential 
 

\end{verbatim}

\newpage
\subsection{cirDefaultValues}
\label{sec:cirDefaultValues}
\index{cirDefaultValues@\textsf{cirDefaultValues}}
\begin{verbatim}
 CIRDEFAULTVALUES lists the default values of the fields that are initialized 
 when a circuit element is defined.
 
     LIST = cirDefaultValues(CIRCUIT, ELTYPE) returns a cell array with the
     default values to which the fields of a circuit element of type ELTYPE
     are initialized when this circuit element is defined with cirElementDef.
 
     See also cirInit, cirElementDef, cirDefaultNames. 
 
   (c) IMEC, 2004
   IMEC confidential 
 

\end{verbatim}

\newpage
\subsection{cirDesignkitInit}
\label{sec:cirDesignkitInit}
\index{cirDesignkitInit@\textsf{cirDesignkitInit}}
\begin{verbatim}
  The field printParamTuple is a cell array that is constructed
  as follows. It consists of groups of three elements. The first element of
  each group is a string that corresponds to a circuit element type. The second
  element contains a list of strings that correspond to the parameters that
  can be printed to or modified in a circuit simulation file. The third
  element is a list of the same length as the second element. Every element
  in this list is a string that is used in the simulation file for the value
  of the corresponding parameter of the first element. For example, for a
  resistor, the first element is 'res'. If now the second element is {'val'}
  and  the third element is {'r'}, then the field "val" of
  a resistor is represented by "r" (this corresponds to the Spectre syntax).
  The possible strings in the second elements of each trio of this list should
  belong to the fields circuit.allowedPrintFields.res,
  circuit.allowedPrintFields.cap, ... 
 
   (c) IMEC, 2004
   IMEC confidential 
 

\end{verbatim}

\newpage
\subsection{cirElementCopy}
\label{sec:cirElementCopy}
\index{cirElementCopy@\textsf{cirElementCopy}}
\begin{verbatim}
 CIRELEMENTCOPY copies a circuit element to another one.
 
     DESTINATION = cirElementCopy(SOURCE, DESTINATION) copies a circuit
     element SOURCE to another circuit element DESTINATION. By this copy
     operation, all fields except for the name field are copied from the
     SOURCE element to the DESTINATION element. It is required that the
     element types of SOURCE and DESTINATION correspond. 
 
 
     EXAMPLES:
 
     Mn1b = cirElementCopy(Mn1a, Mn1b);
 
 
   (c) IMEC, 2004
   IMEC confidential 
 

\end{verbatim}

\newpage
\subsection{cirElementDef}
\label{sec:cirElementDef}
\index{cirElementDef@\textsf{cirElementDef}}
\begin{verbatim}
 CIRELEMENTDEF definition of a circuit element.
 
     CIRCUIT = cirElementDef(CIRCUIT, ELTYPE, ELEMENTNAME) adds a circuit
     element to the datastructure of the given CIRCUIT. This function changes
     the datastructure of the given circuit, which is returned. The circuit
     element needs to be specified by a name ELEMENTNAME, which should be a
     string and by its element type ELTYPE, which should also be a string.
     The circuit element is added as a field to the circuit, such as
     CIRCUIT.ELEMENTNAME. This is in turn a structure, which will have
     different fields depending on the element type. Two fields are already
     defined in this function, namely CIRCUIT.ELEMENTNAME.eltype and
     CIRCUIT.ELEMENTNAME.name, which are filled with the strings ELTYPE and
     ELEMENTNAME, respectively.
     Further, the field nElements of the given CIRCUIT is increased with one,
     the field CIRCUIT.(ELTYPELIST) is augmented with the ELEMENTNAME, and the
     field CIRCUIT.n(ELTYPE) is augmented with one. For example, for an n-MOS
     transistor Mn1, eltype is 'nmos', and the field CIRCUIT.nmosList is
     extended with the name 'Mn1', and the field CIRCUIT.nnmos is augmented
     with one. 
 
     For the circuit element itself, the field "parent" is defined and it is
     equal to the name of the circuit. 
 
     In addition to the fields eltype and name, some other fields, that are 
     specific for each element type (see cirDefaultNames(CIRCUIT, ELTYPE)), 
     are already defined and initialized with default values.
     These default values can be overridden during initialization
     by specifying extra arguments to cirElementDef in addition to the three
     arguments CIRCUIT, ELTYPE and ELEMENTNAME. If no extra arguments are
     specified, then default values are used to initialize the fields of 
     cirDefaultNames(CIRCUIT, ELTYPE). Else, when default values need to be 
     overridden, the syntax is as follows:
     CIRCUIT = cirElementDef(CIRCUIT, ELTYPE, ELEMENTNAME, 'fieldname', value,
     'fieldname', value, ...)
     This means that the fieldname (string) has to be specified as an argument,
     immediately followed by its value as the next argument. 
 
     See also cirElementsCheckIn, cirDefaultNames, cirDefaultValues.
     
     EXAMPLES:
 
     cir = cirElementDef(cir, 'nmos', 'Mn1')
 
          This example defines n-MOS transistor Mn1 and takes default values
          for the fields lg, w, table, paramFields, paramPrintNames, lsos,
          lsod, mult, lsogs, lsogd, mult.
 
     cir = cirElementDef(cir, 'nmos', 'Mn1', 'lg', 2e-6)
 
         This example defines n-MOS transistor Mn1. The channel length lg is
         initialized to 2 micrometer, while the other fields w, table, 
         paramFields, paramPrintNames, lsos, lsod, mult, lsogs, lsogd, mult 
         are initialized to their default value.
 
    cir = cirElementDef(cir, 'nmos', 'Mn3', 'paramFields', {nFingers})
 
        This example defines n-MOS transistor Mn3. The field
        "paramField" is initialized with {nFingers}. This means that only
        the value field "nFingers" will be printed to the circuit simulation
        file. The other fields are initialized with their default value.
 
   cir = cirElementDef(cir, 'nmos', Mn4, 'paramFields', {gm, lg, w})
        This will give an error, since "gm" is not a parameter that can be
        specified as one that will be printed to the circuit simulation file.
 
 
   (c) IMEC, 2004
   IMEC confidential 
 

\end{verbatim}

\newpage
\subsection{cirElementOfType}
\label{sec:cirElementOfType}
\index{cirElementOfType@\textsf{cirElementOfType}}
\begin{verbatim}
 CIRELEMENTOFTYPE returns a list of elements of a given type.
 
     ELEMENTLIST = cirElementOfType(CIRCUIT, ELTYPE) returns a cell array of
     circuit elements of a given type ELTYPE (specified as a string), that
     have been defined in the given CIRCUIT.
 
     See also cirInit, cirDefaultNames, cirDefaultValues.
     
     EXAMPLES:
 
     nmosList = cirElementOfType(cir, 'nmos')
 
 
   (c) IMEC, 2004
   IMEC confidential 
 

\end{verbatim}

\newpage
\subsection{cirElementsCheckIn}
\label{sec:cirElementsCheckIn}
\index{cirElementsCheckIn@\textsf{cirElementsCheckIn}}
\begin{verbatim}
 Definition of circuit elements for a given circuit.
 
   CIRELEMENTSCHECKIN(CIRCUIT, ELEMENTLIST, TOP) defines in the given CIRCUIT
   for a circuit element with name '<name>', a field CIRCUIT.<name>. This field
   is initialized as a circuit element of a certain type, using the function
   cirElementDef. The following types of circuit elements are allowed:
   voltage sources (type "v"), current sources (type "i"), n-MOS transistors
   (type "nmos"), p-MOS transistors (type "pmos"), capacitors (type "cap"),
   resistors (type "res"), inductors (type "ind") and subcircuits (type
   "subckt"). The circuit elements must be specified with the argument
   ELEMENTLIST, which is a structure with the following fields:
   - if independent voltage sources must be defined, then ELEMENTLIST should
     have a field v. If so, then ELEMENTLIST.v should be a cell array, that
     contains the names of all independent voltage sources. If no independent
     voltage sources have to be defined, then either the field ELEMENTLIST.v is
     not defined, or it is equal to the empty cell array {}; 
   - if independent current sources must be defined, then ELEMENTLIST should
     have a field i. If so, then ELEMENTLIST.i should be a cell array, that
     contains the names of all independent current sources. If no independent
     current sources have to be defined, then either the field ELEMENTLIST.i is
     not defined, or it is equal to the empty cell array {};
   - if n-MOS transistors must be defined, then ELEMENTLIST should have a field
     nmos. If so, then ELEMENTLIST.nmos should be a cell array, that
     contains the names of all n-MOS transistors. If no n-MOS transistors have
     to be defined, then either the field ELEMENTLIST.nmos is not defined, or
     it is equal to the empty cell array {}; 
   - idem for the circuit elements of type "pmos", "cap", "res", "ind" and
     "subckt".
 
   If the function CIRELEMENTSCHECKIN is called for the top circuit (which is
   always the case for a circuit without hierarchy), then the argument TOP must
   have a nonzero value. In this case, in addition to defining circuit elements
   of the form CIRCUIT.<name>, the variable <name> is also copied to the base
   workspace. This has been provided for easier manipulation of the circuit
   elements without too much typing overhead and for a better readibility of
   the code. For example for a MOS transistor with name 'Mn1' in the circuit
   with name 'opamp', manipulating the variable Mn1 in the base workspace, is
   more elegant than having to manipulate opamp.Mn1.
   Important remark: if the variable <name> is used for the sizing of the
   circuit, then, after the sizing of <name>, all fields of <name> should be
   copied to CIRCUIT.<name>. This is done with the function
   cirElementsCheckOut. 
 
   See also cirElementDef, cirElementsCheckOut
 
   EXAMPLES: 
 
   Let's define an elementary OTA. The name of the corresponding variable will
   be "elOta". First we define a list of circuit elements, grouped per element
   type: 
      elementList.v = {'Vdd', 'VinN', 'VinP', 'Vbias'};
      elementList.nmos = {'Mn1a', 'Mn1b', 'Mnbias'};
      elementList.pmos = {'Mp1a', 'Mp1b'};
      elementList.i = {};

\end{verbatim}

\newpage
\subsection{cirElementsCheckOut}
\label{sec:cirElementsCheckOut}
\index{cirElementsCheckOut@\textsf{cirElementsCheckOut}}
\begin{verbatim}
 CIRELEMENTSCHECKOUT copies circuit element with name <x> to <circuit>.<x>
 
  See also cirElementsCheckIn
 
   EXAMPLES:
 
  we have a circuit with name 'commSource'. The circuit elements are
  commSource.Mn1, commSource.Rl, commSource.Vdd and commSource.Vin
  With the function cirElementsCheckIn these elements have been copied to the
  variables Mn1, Rl, Vdd and Vin. In the Matlab code that sizes these elements,
  the fields of these elements get meaningful values, e.g. Vin.val gets a
  value, ...
  After this sizing we can use cirElementsCheckOut to copy the fields of Mn1,
  Rl, Vdd and Vin back to commSource.Mn1, commSource.Rl, commSource.Vdd and
  commSource.Vin. This is done by the command
 
  commSource = cirElementsCheckOut(commSource)
 
 
 
   (c) IMEC, 2004
   IMEC confidential 
 

\end{verbatim}

\newpage
\subsection{cirInit}
\label{sec:cirInit}
\index{cirInit@\textsf{cirInit}}
\begin{verbatim}
 CIRINIT initialization of a circuit
 
  circuit = cirInit(NAME, TITLE, HIERARCHY, ELEMENTLIST, ...
    SPECS (optional), DESIGNCHOICES (optional), DESIGNKITNAME (optional),
    SIMULFILE (optional), NMOSTABLE (optional), ... 
    PMOSTABLE (optional), SIMULATOR (optional), ...
    PARAMSIMULSKELFILE (optional)) 
    initializes a datastructure for a circuit. This new datastructure is
    returned. If no hierarchy is built in the circuit, then the arument
    HIERARCHY must have the value 'top'. In a hierarchical circuit, this
    argument must have this value as well. If a subcircuit is initialized, then
    the argument HIERARCHY must have the value 'sub'.
 
    The datastructure of a circuit is a structure with the following fields:
 
      1. name: a string, which is specified as the argument NAME of this
         initialization function.
      
      2. title: a title string, which is specified as the argument TITLE 
         of this initialization function.
 
      3. simulFile: a string that represents the name of the file to which the
         values will be written for the different circuit parameters that are
         determined with Matlab. This string is specified as the argument
         SIMULFILE to this initialization function.
 
      4. simulSkeletonFile: this field is only used for circuit simulations in 
         SmartSpice. It is a string representing the name of the file that
         contains the part of a circuit simulation file that does not contain
         any parameters that are determined in Matlab.  Therefore, it is an optional
         argument SIMULSKELFILE, that does not need to be specified when other 
         simulators are used. 
      
      5. defaults: a structure that contains default values for the different
         circuit elements that are supported. These default values are used when a
         circuit element is initialized. For a nMOS and pMOS transistor,
         default values are used for 
         - channel length (field "lg")
         - channel width (field "w")
         - table of intrinsic operating point parameters (field "table")
         - length of drain/source diffusion areas not between two
          poly stripes (fields "lsod" and "lsos", respectively)
         - length of drain/source diffusion areas between two poly stripes
          (fields "lsogd" and "lsogs", respectively)
         - the amount of fingers (field "nFingers")
         - the multiplicity (field "mult")
         - the parameters that will be written
           to a circuit simulation file (field "paramFields"). This is a cell
           array of strings, each of them being a field of a MOS transistor
           structure (e.g. "lg"). The value of these fields will be determined
           in the Matlab design plan.
         - the printnames that will be used for the parameters in the field
           paramFields of the transistor (field "paramPrintNames"). This is also
           a cell array of strings. This cell array has the same length as the
           cell array "paramFields". The printnames are used for the names of
           the parameters that are written into the circuit simulation
           file. E.g. the printname of the length of a MOS transistor could be
           'l'. In that case, a parameter would be made of the form
           '<mos><glue string>l', e.g. m1_l
      For a voltage source, current source, capacitor, inductor and resistor, 
      default values are used for 
        - value (field "val")
        - field "paramFields" (similar meaning as for a MOS transistor)
        - field "paramPrintNames" (similar meaning as for a MOS transistor)
 
      6. allowedPrintFields: an array of strings (cell array) that contains the
      names of the fields that can be passed to/changed in a circuit simulation
      file. For example, for a MOS transistor this list is
      {'w', 'lg', 'ad', 'as', 'pd', 'ps', 'nFingers', 'lsos', 'lsod', ...
       'lsogs', 'lsogd', 'mult', 'nrs', 'nrd'}.
      For a resistor, capacitor, inductor, voltage source and current source,
      this list contains one element, namely 'val'.
 
      7. nElements: number of circuit elements that already have been defined.
 
      8. fields that each represent a circuit element that belongs to the
         circuit. At initialization there is no such field. During the Matlab
         program these fields are filled. E.g., when an n-MOS transistor m1
         is defined (see function cirElementDef) then there will be a field
         "m1" in the circuit.
         The following types of elements are supported: nmos, pmos, resistor,
         capacitor, voltage source, current source (see function
         cirCheckEltype). Note that not every circuit element that appears in
         the netlist of the circuit, needs to be defined in Matlab. Only those
         elements need to be defined for which some parameters are computed in
         Matlab. For example, if the power supply voltage is fixed, then the
         corresponding voltage source does not need to be added to the
         datastructure. 
     
      9. fields that represent for each circuit element type a list of the
         elements of that type and the length of that list:
         - nmosList and nnmos
         - pmosList and npmos
         - resList and nres, capList and ncap, indList and nind, vList and
           nv, iList and ni, subcktList and nsubckt
 
      10. designkit
 
      11. if the argument SPECS has been specified as an argument that differs
          from 0 or from an empty string, then the circuit that is initialized
          gets a field for its specifications. This field is meant to be a
          structure. This is useful to store with the circuit sizing.
          Example: 
               if a structure "spec" has the following fields
                   spec.gain = 5
                   spec.currentConsumption = 1e-3;
               then the inclusion of "spec" in the argument list of cirInit to
               initialize a circuit with name "amp", will result in the field 
               amp.spec and the value will be
                   amp.spec.gain = 5
                   and
                   amp.spec.currentConsumption = 1e-3;
 
     12. if the argument DESIGNCHOICES has been specified as an argument that differs
          from 0 or from an empty string, then the circuit that is initialized
          gets a field for its design choices. This field is meant to be a
          structure. This is useful to store with the circuit sizing.
          Example: 
               if a structure "choice" has the following fields
                   choice.Mn1.lg = 1e-6
                   choice.Ibias.val = 1e-6
               then the inclusion of "choice" in the argument list of cirInit to
               initialize a circuit with name "amp", will result in the field 
               amp.choice and the value will be
                   amp.choice.Mn1.lg = 1e-6
                   and
                   amp.choice.Ibias.val = 1e-6
              
 
  See also cirSubcktInit, cirElementDef, cirElementOfType, cirCheckElType,
  cirSupportedEltypes, cirDesignkitInit 
 
 
   (c) IMEC, 2004
   IMEC confidential 
 

\end{verbatim}

\newpage
\subsection{cirMosCheckSaturation}
\label{sec:cirMosCheckSaturation}
\index{cirMosCheckSaturation@\textsf{cirMosCheckSaturation}}
\begin{verbatim}
  checking all MOS transistors for saturation
 
  CIRMOSCHECKSATURATION(CIRCUIT, FID) checks all n-MOS and p-MOS transistors of
  a circuit whether or not they are saturated. The function writes a warning to
  the file identifier FID for every transistor that is not saturated
 
   (c) IMEC, 2004
   IMEC confidential 
 

\end{verbatim}

\newpage
\subsection{cirPrintHeader}
\label{sec:cirPrintHeader}
\index{cirPrintHeader@\textsf{cirPrintHeader}}
\begin{verbatim}
 CIRPRINTHEADER printing of a header for a circuit that is going to be sized
 
   CIRPRINTHEADER(CIRCUIT) prints a header for a circuit to the
   screen. This header contains the title of the circuit, the actual date and
   time when this function is called, and the name of the user who is calling
   that function.
   
   CIRPRINTHEADER(CIRCUIT, FID) prints the header to the file
   associated with file identifier FID. FID is an integer file identifier
   obtained from FOPEN. It can also be 1 for standard output (the screen) or 2
   for standard error.
 
   CIRPRINTHEADER(CIRCUIT, FID, SPECS) prints in addition the
   set of specs of the circuit. The variable SPECS must be a structure. All
   fields of this structure are printed.
 
  CIRPRINTHEADER(CIRCUIT, FID, SPECS, CHOICES) prints in
  addition the design choices that are made during the sizing of the circuit. 
   The variable CHOICES must be a structure. All fields of this structure are
   printed. 
 
 
   See also FOPEN.
 
 
   (c) IMEC, 2004
   IMEC confidential 
 

\end{verbatim}

\newpage
\subsection{cirSimulFile}
\label{sec:cirSimulFile}
\index{cirSimulFile@\textsf{cirSimulFile}}
\begin{verbatim}
 CIRSIMULFILE retrieves the circuit simulation file
 
    FILENAME = cirSimulFile(CIRCUIT) returns a string that contains for the
    given CIRCUIT the name of the file that will be used to verify the design
    with Matlab of this CIRCUIT.
 
 
 
   (c) IMEC, 2004
   IMEC confidential 
 

\end{verbatim}

\newpage
\subsection{cirSimulSkelFile}
\label{sec:cirSimulSkelFile}
\index{cirSimulSkelFile@\textsf{cirSimulSkelFile}}
\begin{verbatim}
 CIRSIMULSKELFILE retrieves the circuit simulation file
 
    FILENAME = cirSimulSkelFile(CIRCUIT) returns a string that contains for the
    given CIRCUIT the name of the file that contains the part of the circuit
    simulation input file that is fixed (i.e. contains no elements that are
    determined with Matlab).
 
 
 
   (c) IMEC, 2004
   IMEC confidential 
 

\end{verbatim}

\newpage
\subsection{cirSupportedEltypes}
\label{sec:cirSupportedEltypes}
\index{cirSupportedEltypes@\textsf{cirSupportedEltypes}}
\begin{verbatim}
 CIRSUPPORTEDELTYPES lists the element types that are supported in the set of
   Matlab functions for circuit sizing
 
   RESULT = CIRSUPPORTEDELTYPES returns a cell array that contains the element
   types that are supported.
 
   (c) IMEC, 2004
   IMEC confidential 
 

\end{verbatim}

\newpage
\subsection{cirWriteParam}
\label{sec:cirWriteParam}
\index{cirWriteParam@\textsf{cirWriteParam}}
\begin{verbatim}
 CIRWRITEPARAM write a parameter statement for the given simulator
 
     cirWriteParam(FID, SIMULATOR, PARAM, VALUE) writes a parameter
     statement with the appropriate syntax for the given SIMULATOR for the
     given PARAMETER (should be a string). The VALUE of that parameter
     should also be given as an argument. The parameter statement is
     written to the given file identifier FID
 
     See also fopen, cirWriteParams
 
 
 
   (c) IMEC, 2004
   IMEC confidential 
 

\end{verbatim}

\newpage
\subsection{cirWriteParams}
\label{sec:cirWriteParams}
\index{cirWriteParams@\textsf{cirWriteParams}}
\begin{verbatim}
 CIRWRITEPARAMS writes parameters that have been determined in Matlab to a
 circuit simulation file for a given SIMULATOR.
  
    Supported simulators are smartspice, hspice, spectre
 
 
    For smartspice we put everything into one file in order to avoid .include
    statements, which seem to be buggy in smartspice. Therefore, we dump the
    parameters in a temporary file, and in the end we combine this file and
    the skeleton file into the simulation file.
 
    For HSPICE, the simulation file is just a set of lines with .param
    statements. The skeleton file is not used here.
 
    For spectre, we replace in a circuit netlist the values of the circuit
    elements in the top circuit and in the subcircuits. The skeleton file is
    not used neither here.
 
   See also cirWriteParamsSpice and cirWriteParamsSpectre
 
 
   (c) IMEC, 2004
   IMEC confidential 
 

\end{verbatim}

\newpage
\subsection{cirWriteParamsSpectre}
\label{sec:cirWriteParamsSpectre}
\index{cirWriteParamsSpectre@\textsf{cirWriteParamsSpectre}}
\begin{verbatim}
 CIRWRITEPARAMS writes parameters that have been determined in Matlab to a
 circuit simulation file for spectre. We replace
    in a circuit netlist the values of the circuit elements in the top circuit
    and in the subcircuits. We make use of the function scsSetPar. This
    function performs a substitution in the simulation file circuit.simulFile,
    using a structure of parameters. This structure is constructed in this
    function cirWriteParamsSpectre. The structure can be saved in a .mat file,
    which can be given as a second argument.
 
    See also scsSetPar
 
   EXAMPLE:
  
   cirWriteParamsSpectre(circuit)
   Here the substitution is made without saving the parameter structure.
 
   cirWriteParamsSpectre(circuit, 'parStruct.mat')
   Here the parameter structure is saved to a file parStruct.mat
 
 
   (c) IMEC, 2004
   IMEC confidential 
 

\end{verbatim}

\newpage
\subsection{cirWriteParamsSpice}
\label{sec:cirWriteParamsSpice}
\index{cirWriteParamsSpice@\textsf{cirWriteParamsSpice}}
\begin{verbatim}
 CIRWRITEPARAMSSPICE writes parameters that have been determined in Matlab to 
  a circuit simulation file for Spice. Supported Spice versions are  
   smartspice and hspice. 
 
 
    For smartspice we put everything into one file in order to avoid .include
    statements, which seem to be buggy in smartspice. Therefore, we dump the
    parameters in a temporary file, and in the end we combine this file and
    the skeleton file into the simulation file.
    For HSPICE, the simulation file is just a set of lines with .param
    statements. The skeleton file is not used here.
 
   (c) IMEC, 2004
   IMEC confidential 
 

\end{verbatim}

\newpage
\newpage
